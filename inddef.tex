\documentclass[11pt,a4paper]{ltjsarticle} % lualatex
%
\usepackage{fontspec}
%
\usepackage{amsmath,amssymb} % 数式用
\usepackage{amsthm} % 定理環境
\usepackage{braket} % カッコ
\usepackage{prettyref} % 相互参照
%
%%%%======prettyref===========%%%%%%%%%
%
\newrefformat{eq}{式{\eqref{#1}}}
\newrefformat{thm}{定理{\ref{#1}}}
\newrefformat{ex}{例{\ref{#1}}}
\newrefformat{lemma}{補題{\ref{#1}}}
\newrefformat{que}{問{\ref{#1}}}
%
%%%%%%======biblatex=============%%%%%%%%%%
%
\usepackage[% 
backend=biber,
url=true,
doi=false,
eprint=false,
isbn=true,
style=phys
]{biblatex} % use "biblatex"
\DeclareSortingTemplate{mysorting}{
  \sort{\citeorder}
  \sort{\field{sotname} \field{author} \field{translator} \field{editor}}
  \sort{\field{year}}
  \sort{\field{title}}
}
%
\DeclareFieldFormat[article]{journaltitle}{\mkbibemph{#1}}
\DeclareFieldFormat[book]{title}{\mkbibemph{#1}}
\ExecuteBibliographyOptions{sorting=mysorting}
%
\addbibresource{bib/youryou.bib}
%
%%%%%%%%%%%%%%========hyperref============%%%%%%%%%%
%
\usepackage[unicode]{hyperref} % hyperlink
\hypersetup{% setting hyperref
  bookmarksnumbered=true,
  bookmarksopen=true,
  bookmarkstype=toc,
  pdfborder={0 0 0},
  colorlinks=true
}
%
%%%%%%==========定理環境のカスタマイズ===============%%%%%%%%%
%
\newtheoremstyle{mystyle} % 定理環境のスタイルを作成
{} % space above
{} % space below
{\normalfont} % body font 
{} % indent amount
{\bfseries} % theorem head font
{} % punctuation after theorem head
{10pt} % space after theorem head
{\thmname{#1}\thmnumber{#2}\thmnote{\textbf{\hspace{1pt}(#3)}}} % theorem head spec
\theoremstyle{mystyle} % 作成したスタイルを使用
%
\newtheorem{thm}{定理} % \begin{thm} で定理を書く
\newtheorem{lemma}{補題} % \begin{lemma} で補題
\newtheorem*{coro}{系}  % \begin{coro} で系
\newtheorem{que}{問} % \begin{que} で問
\newtheorem*{ans}{[解答]} % \begin{ans}で解答
%
%
%%%%%================proof環境のカスタマイズ========================%%%%%%%%%%%
\makeatletter
\renewenvironment{proof}[1][\proofname]{\par
  \pushQED{\qed}%
  \normalfont \topsep6\p@\@plus6\p@\relax
  \trivlist
\item[\hskip\labelsep
  \itshape
  %    #1\@addpunct{.}]\ignorespaces% DELETED
  #1]\ignorespaces% ADDED
  }{%
  \popQED\endtrivlist\@endpefalse
}
\makeatother
%                       ドットが消える
\renewcommand{\proofname}{\textbf{[証明]}}
%
%%%%%%%%%%%%==========タイトル===============%%%%%
\title{漸化式の解を推測するアレがアレな件}
\author{野口 匠}
\date{2018/12/22}
\begin{document}
%
\maketitle
%
漸化式によって帰納的に定義された数列の一般項を求める
という問題は,大学受験レベルの問題でよくあるものである.
その中で,「一般項を推測して帰納法で示す」というものがあった.
高校の数学Bの教科書を見てみると,次のような問題が記載されている.
\begin{que} \label{que:suisoku}
  (ここに引用する問題が入る)
\end{que}
この問題に対する解答は次のようになっている.
\begin{ans}
  ここに解答
\end{ans}

定式化して証明を述べておこう.
\begin{thm} \label{thm:inddef}
  空でない集合$X$と写像$G \colon \bigcup_{n \in \mathbb{N}} 
  X^{\mathbb{N} \langle n \rangle} \longrightarrow X$
  が与えられたとする.
  このとき,$x_0 \in X$を1つ定めるごとに,
  写像$f \colon \mathbb{N} \longrightarrow X$で
  \begin{align}
    \begin{aligned}
      f(0) & = x_0 \\
      f(n) & = G \left( f|_{\mathbb{N} \langle n \rangle} \right)
      \quad ( n \in \mathbb{N} - \Set{0} )
    \end{aligned}
    \label{eq:inddef}
  \end{align}
  を満たすものが一意に定まる.
\end{thm}

\begin{proof}
  先に一意性の方から示しておこう.
  写像$f,g \colon \mathbb{N} \longrightarrow X$で
  \prettyref{eq:inddef}を満たすようなものを任意にとり,
  すべての$n \in \mathbb{N}$に対して$f(n) = g(n)$が成り立つことを
  帰納法で示す.
  $f(0) = x_0 = g(0)$より$f(0)=g(0)$である.
  各$n \in \mathbb{N}$に対し,$i= 0,1, \ldots, n$に対して
  $f(i)=g(i)$であるとすると,
  $f|_{\mathbb{N} \langle n+1 \rangle} = g|_{\mathbb{N} \langle n+1 \rangle} $
  だから$f(n+1) = G \left( f|_{\mathbb{N} \langle n+1 \rangle} \right) =
  G \left ( g |_{\mathbb{N} \langle n+1 \rangle} \right) = g(n+1)$
  より$f(n+1) =g(n+1)$となる.
  ゆえに$f(n) = g(n)$となるから,
  すべての$n \in \mathbb{N}$に対して$f(n) = g(n)$となり,
  $f=g$を得る.

  \prettyref{eq:inddef}を満たす写像の存在を示そう.
  自然数$n$の条件$P(n)$を
  「写像$f \colon \mathbb{N} \langle n \rangle \longrightarrow X$で
  すべての$m \in \mathbb{N} \langle n \rangle$に対して
  $f(m) = G \left( f|_{\mathbb{N} \langle m \rangle } \right)$
  を満たすものが存在する」と定める.
  一意性の証明とまったく同様にして,
  各$n \in \mathbb{N}$に対して$P(n)$を成り立たせるような写像$f$は
  たかだか1つであることが示される.その写像を$f_n$と表記する.
  すべての自然数$n$に対して$P(n)$が成り立つことを帰納法によって示そう.
  $\mathbb{N} \langle 0 \rangle = \varnothing$だから,
  $P(0)$を成り立たせるような写像$f$として空写像がとれる.従って$P(0)$は成り立つ.
  各$n \in \mathbb{N}$に対し,$i=0,1,\ldots,n$に対して
  $P(i)$が成り立つと仮定する.
  いま,写像$f \colon \mathbb{N} \langle n+1 \rangle \longrightarrow X$を
  \begin{align*}
    f (m) = 
    \begin{cases}
      G \left( f_{m} \right) & ( \text{$m=n$のとき} ) , \\
      f_{m+1} (m) & (\text{それ以外のとき})
    \end{cases}
  \end{align*}
  と定めると,この$f$はすべての$m \in \mathbb{N} \langle n+1 \rangle$
  に対して$f(m) = G \left( f|_{\mathbb{N} \langle m \rangle} \right)$
  を満たす.ゆえに$P(n+1)$も成り立つから,
  すべての自然数$n$に対して$P(n)$が成り立つ.
  そこで,写像$f \colon \mathbb{N} \longrightarrow X$を
  \begin{align*}
    f(n) = f_{n+1} (n) \quad ( n \in \mathbb{N} )
  \end{align*}
  と定めると,この$f$は\prettyref{eq:inddef}を満たす.
\end{proof}

\prettyref{thm:inddef}において,各
$\varPhi \in \bigcup_{n \in \mathbb{N}} X^{\mathbb{N} \langle n \rangle}$
に対して$\varPhi \in X^{ \mathbb{N} \langle i \rangle}$となる

%
\printbibliography[title=参考文献]
\end{document}
